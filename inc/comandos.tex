% Comandos personalizados

% \fechaPresentacion :: para escribir la fecha de presentación del trabajo
\newcommand{\fechaPresentacion}{\today}
% \unlp :: para escribir "Universidad Nacional de La Plata"
\newcommand{\unlp}{Universidad Nacional de La Plata}
% \facultad :: para escribir "Facultad de Informática"
\newcommand{\facultad}{Facultad de Informática}
% \cespi :: para escribir "CeSPI"
\newcommand{\cespi}{CeSPI}
% \direccionDesarrollo :: Para escribir "Dirección de Desarrollo del CeSPI"
\newcommand{\direccionDesarrollo}{Dirección de Desarrollo del \cespi\xspace}
% \tituloTrabajo :: Para escribir el título de la tesina
\newcommand{\tituloTrabajo}{Propuesta de rediseño de la nube de servicios de la UNLP}
% \miguelcarbone :: para escribir "Miguel Carbone"
\newcommand{\miguelcarbone}{Miguel Carbone}
% \carbonemiguel :: para escribir "Carbone, Miguel"
\newcommand{\carbonemiguel}{Carbone, Miguel}
% \nahuelcuesta :: para escribir "José Nahuel Cuesta Luengo"
\newcommand{\nahuelcuesta}{José Nahuel Cuesta Luengo}
% \cuestanahuel :: para escribir "Cuesta Luengo, José Nahuel"
\newcommand{\cuestanahuel}{Cuesta Luengo, José Nahuel}

% \eng{English expression} :: para denotar que "English expression" está en inglés
\newcommand{\eng}[1]{\textit{#1}}

% \caratula :: para generar la carátula de la tesina
\newcommand{\caratula}{
  \begin{center}
    \huge{\unlp}\\
    \vspace{5mm}
    \large{\facultad}\\
    \vspace{5mm}
    \large{Tesina de la Licenciatura}\\
    \vspace{15mm}
    \huge{\tituloTrabajo}\\
    \vspace{10mm}
    \large{\textbf{\carbonemiguel} \\ \textbf{\cuestanahuel}}\\
    \vspace{30mm}
    \large{Directora: Banchoff Tzancoff, Claudia}\\
    \large{Co-Directora: Queiruga, Claudia}\\
    \large{Asesor Profesional: Rodriguez, Christian Adrián}\\
    \vspace{20mm}
    \normalsize{\fechaPresentacion}\\
  \end{center}
}

% \icode{texto preformateado} :: para imprimir como fixed-width "texto preformateado" (útil para código inline)
\newcommand{\icode}[1]{\begin{verbatim}#1\end{verbatim}}
