\todo{Escribir una breve introducción para el trabajo}

Una de las grandes problemáticas que encontramos a diario en nuestro trabajo como desarrolladores de sistemas informáticos en la \direccionDesarrollo, \unlp, es el uso y mantenimiento de la nube de servicios que nuestra Dirección ha implementado hace ya más de cuatro años, y que de a poco se ha ido convirtiendo en un obstáculo que retrasa el avance de mejores soluciones integrales. 

Nuestro equipo de trabajo se compone de más de una decena de integrantes, entre las cuales nos repartimos los distintos desarrollos que tenemos. En particular, nosotros (Nahuel y Miguel) nos hemos dedicado a investigar soluciones tecnológicas y a realizar el análisis que aquí presentamos con el fin de, en una etapa posterior, continuar con el resto del proceso de rediseño de la nube de servicios, incorporando a otros integrantes del equipo en estas tareas. De manera similar nos hemos propuesto realizar el desarrollo del caso testigo, para luego transmitir la experiencia al resto del equipo.

La implementación actual presenta importantes falencias que dificultan su mantenimiento e incorporación de nuevos servicios. Tanto es así, que ante la necesidad de brindar un nuevo servicio, éste se desarrolla integrado (\textit{dentro}) a la aplicación que produce la información, lo cual genera un alto grado de acoplamiento. Esto se debe a que la arquitectura actual de la nube no permite la incorporación de nuevos servicios que estén fuera de la aplicación monolítica que brinda sus servicios. Entre otras problemáticas que desarrollaremos con mayor detalle en los siguientes capítulos, podemos enumerar la falta de un protocolo estándar de peticiones y respuestas para el acceso a los datos, las inconsistencias presentes, falta de documentación, desactualización tecnológica y falta de un diseño pensado para la escalabilidad.

En esta tesina analizaremos de manera crítica el estado actual de la implementación de la nube de servicios y abordaremos sus problemáticas con una propuesta basada en un enfoque más actualizado, bien fundado y planificado, pensado para adaptarse al constante cambio y crecimiento de los servicios a brindar. Somos conscientes que una capa dinámica de servicios debidamente planificada es crítica en el ecosistema de aplicaciones que desarrollamos en nuestra Dirección y una necesidad impostergable; y es en ese sentido que planteamos la temática para el presente trabajo, el cual será el punto de partida para una reimplementación completa de esta nube de servicios.
