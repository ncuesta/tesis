Como punto de partida en nuestro análisis comenzaremos por recapitular la evolución histórica de la nube de servicios de la \unlp, de manera tal que pueda comprenderse cómo el crecimiento del conjunto de aplicaciones que desarrollamos fue afectando el enfoque tomado y llevándonos a replantear el diseño que hasta ese momento tenían las fuentes de datos compartidos entre esas aplicaciones para adaptarse a nuevas necesidades.

Este capítulo inicial tiene como objetivo dos puntos principales: contextualizar al lector en el dominio del presente trabajo y, aprovechando ese desglose lógico que haremos para explicar la composición de la nube de servicios, analizar las falencias y los problemas que en ella existen. A partir de las conclusiones de este capítulo, desarrollaremos en los subsiguientes nuestra propuesta para el nuevo diseño de este concentrador de servicios, elemento crítico para nuestras aplicaciones.
