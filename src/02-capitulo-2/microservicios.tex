\subsection{La evolución de SOA: Microservicios}
\label{microservicios}

Como vimos en el apartado anterior, \gls{acro:soa} define qué principios debe cumplir un sistema distribuido para satisfacer los objetivos de la organización, que incluyen, facilidad y flexibilidad en la integración de sistemas \eng{legacy}, reducción en los costos de implementación y ágil adaptación a los cambios. Si bien esto ya establece precondiciones para un diseño, \gls{acro:soa} en sí no es un patrón concreto. Por el contrario, y similarmente a otros diseños de arquitectura de sistemas distribuidos como REST (el cual será tratado en la \autoref{standard:rest}), \gls{acro:soa} es un conjunto de propiedades que dichos sistemas debieran cumplir y a partir de los cuales podemos observar características comunes entre aquellos que sigan ese tipo de arquitecturas.

Es sobre la base de características de un tipo de sistemas como \gls{acro:soa} que se construyen luego los patrones de diseño. A modo de referencia se puede consultar el libro \eng{``SOA Patterns''} donde se definen más de 20 patrones orientados a servicios, como \eng{Service Host}\cite[p.~19]{soapatterns}, \eng{Composite Front End (Portal)}\cite[p.~148]{soapatterns} o \eng{Service Bus}\cite[p.162]{soapatterns}. Todos estos (y otros) patrones obedecen a los principios de las Arquitecturas Orientadas a Servicios, pero ninguno \textit{es} \gls{acro:soa} en sí mismo, ni \gls{acro:soa} \textit{es} solo uno de estos patrones. De hecho, si analizamos el caso particular de estos patrones podemos observar que inclusive pueden combinarse entre sí: el primero (\eng{Service Host}) guía en cómo se puede manejar el ambiente en que corren los servicios y los clientes de los mismos para simplificar su gestión, el segundo (\eng{Portal}) habla de cómo proveer una interfaz que combine más de un servicio, y el tercero (\eng{Service Bus}) propone la inclusión de un canal de comunicación previo a los servicios para gestionar las peticiones y respuestas con mejoras sobre la autogestión de peticiones por parte de los propios servicios.

En este contexto es que queremos centrarnos en un patrón de diseño de sistemas distribuidos emergente que se plantea como una alternativa al desarrollo de aplicaciones monolíticas.
Las aplicaciones monolíticas, típicamente consisten en componentes fuertemente acoplados, que son parte de una única unidad \eng{deployable}, resultando incómodo y dificultosa la incorporación de cambios, \eng{testing} y \eng{deployment} de la aplicación.  Es por esto que podemos encontrar ciclos de \eng{deployment} mensual (\eng{monthly deploymenbt}) en grandes organizaciones de IT.

El patrón de microservicios, trata estas cuestiones, separando la aplicación en múltiples unidades \eng{deployables} (\eng{service components}), que pueden ser desarrolladas, testeadas y deployadas independientemente de otros \eng{service component}. \todo{quizás tengamos que hablar mas de service component}.\cite[p.~27]{richards2015}

\begin{figure}
  \includegraphics[width=\linewidth]{src/images/02-capitulo-2/basic_microservices_arquitecture_pattern.png}
  \caption{Arquitectura básica de microservicios}
  \label{fig:basic_microservices_arquitecture_pattern}
\end{figure}
