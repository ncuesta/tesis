\subsection{Arquitecturas Orientadas a Servicios}
\label{soa}

La Arquitectura Orientada a Servicios (\gls{acro:soa}) establece un marco de diseño para la integración de aplicaciones independientes permitiendo acceder desde la red a sus funcionalidades, las cuales se ofrecen como servicio.  Habitualmente \gls{acro:soa} es implementado mediante Servicios Web, tecnología basada en estándares e independiente de la plataforma que provee los datos, de esta manera \gls{acro:soa} puede descomponer las aplicaciones monolíticas en un conjunto de servicios.  Existen varias definiciones de \gls{acro:soa}, muchas incluyen el término Web Service, pero \gls{acro:soa} y Web Service no son lo mismo.  \gls{acro:soa} es un paradigma y Web Service es una forma posible de implementar \gls{acro:soa}.

Según Thomas Erl, \gls{acro:soa} establece un modelo arquitectónico que tiene como objetivo mejorar la eficiencia, agilidad y productividad de una organización mediante la colocación de los servicios como el principal medio a través del cual se realizan los objetivos estratégicos asociados a la comutación orientada a servicios.

Para el Modelo de Referencia \gls{acro:oasis}, \gls{acro:soa} es un paradigma para organizar y utilizar capacidades distribuidas que pueden estar bajo el control de dominios diferentes.

\gls{acro:soa} incluye prácticas y procesos que se basan en el hecho de que los sistemas distribuidos no son controlados por los mismos propietarios. Diferentes equipos, departamentos, o incluso diferentes organizaciones pueden gestionar diferentes sistemas. Este concepto es clave para entender \gls{acro:soa} y grandes sistemas distribuidos.

En el pasado, se han propuesto una gran cantidad de métodos para resolver el problema de la integración de sistemas distribuidos mediante la eliminación de la heterogeneidad, pero sabemos que estos enfoques no funcionan. Los grandes sistemas distribuidos con diferentes propietarios son heterogéneos  El enfoque \gls{acro:soa} acepta esta heterogeneidad, de manera similar a los métodos ``ágiles'' de desarrollo de software, que aceptan que los requisitos cambian en lugar de tratar de luchar contra esos cambios, \gls{acro:soa} acepta que existe una gran heterogeneidad en los grandes sistemas. No se puede introducir \gls{acro:soa} diseñando todo desde cero. Hay que lidiar con el hecho de que la mayoría de los sistemas legados que se encuentran en producción, se mantendrán.
