Siendo la parte de la arquitectura que mayor desarrollo y mantenimiento de nuestra parte implicará, decidimos utilizar el lenguaje Ruby para implementar los servicios web por los siguientes motivos:

\begin{itemize}
  \item \textbf{Robustez sin sacrificio de la elegancia:} Con varias aplicaciones en producción desarrolladas en este lenguaje, hemos comprobado que es robusto y potente, pero también elegante y \textit{cómodo} de utilizar.
  \item \textbf{Agilidad:} Implementar aplicaciones en Ruby, dada la gran oferta de librerías y frameworks que posee, es realmente ágil y práctico.
  \item \textbf{Nuestra experiencia:} Hace más de 3 años que desarrollamos a diario aplicaciones, herramientas y scripts con este lenguaje, y en estos años no hemos hecho más que apreciar cada vez más las bondades que tiene.
\end{itemize}

Partiendo de esa base, tomamos los dos frameworks para desarrollo de aplicaciones web más consolidados y utilizados\footnote{Para realizar esta apreciación nos basamos en las investigaciones que hemos hecho en el pasado como parte de nuestro trabajo en el \cespi y en los datos que el sitio \eng{The Ruby Toolbox} provee: \url{https://www.ruby-toolbox.com/categories/web_app_frameworks}.} en el ámbito de Ruby como posibles opciones con las cuales implementar nuestros servicios web: Ruby on Rails y Sinatra.

En este apartado describiremos ambos frameworks comparativamente para concluir en la elección que hemos hecho para el presente trabajo.
