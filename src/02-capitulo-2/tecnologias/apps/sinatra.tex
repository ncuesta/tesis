\paragraph{Sinatra}
\label{soa:tecnologias:sinatra}

Sinatra, más que un framework, es un \gls{acro:dsl} simple para el desarrollo rápido de aplicaciones web. Comparado con un framework completo como \nameref{soa:tecnologias:rails}, Sinatra ofrece un conjunto reducido de funcionalidad, el estrictamente necesario para implementar la vista y el controlador del patrón \gls{acro:mvc}, pero no soporta oficialmente una forma concisa de organizar la lógica de negocios de las aplicaciones (la \textit{M} de \gls{acro:mvc}).

Su mayor fortaleza radica en que permite rápidamente implementar aplicaciones web sencillas agregando un \eng{overhead} mínimo. Podemos implementar una aplicación que reciba peticiones y responda con documentos \gls{lang:html} o \gls{lang:json} en poco tiempo y con muy pocas líneas de código, pero a medida que la complejidad crece es necesario complementar Sinatra con otras librerías de manera manual, lo cual no hace de esta herramienta un candidato fuerte para solucionar nuestras necesidades al implementar los servicios de la nube de la \unlp. Por ejemplo, si deseamos implementar modelos con conexión a una base de datos seguramente agreguemos \texttt{ActiveRecord}, la capa de abstracción de \nameref{soa:tecnologias:rails}, y en el proceso deberemos manejar nosotros mismos la configuración y lógica de conexión a esa base de datos; similarmente ocurrirá si necesitamos utilizar una base de datos en memoria para el caching de nuestras aplicaciones (como podría ser \gls{db:redis} o \gls{db:memcached}).

\subparagraph{Sencillez compleja}

En el pasado hemos utilizado Sinatra para implementar algunas aplicaciones web sencillas, principalmente orientadas a proveer \glspl{acro:api} que den soporte a clientes para mostrar datos a los usuarios. En esa experiencia comprobamos lo sencillo que es implementar aplicaciones pequeñas con esta herramienta, lo cual es altamente beneficioso para desarrolladores experimentados así como para aquellos que no lo son. Tiene un conjunto claro de directivas, una documentación extensa y detallada traducida al español, y promueve el desarrollo ágil y conciso.

También nos encontramos con los inconvenientes que trae aparejado Sinatra a la hora de salir de la \textit{zona de comfort} de la herramienta: como mencionamos anteriormente, el trabajo adicional que requiere utilizar bases de datos, enviar correos electrónicos o implementar técnicas de caching, es un \eng{overhead} no funcional que afecta en gran medida el desarrollo de las aplicaciones, quitando tiempo útil para destinarlo en tomar decisiones triviales que no afectan funcionalmente a las aplicaciones. Al ser tan sencillo, Sinatra tampoco promueve una forma estándar de organizar el código, lo cual es propenso a decisiones tendenciosas por parte de cada desarrollador a la hora de ubicar los componentes lógicos de la aplicación, resultando en muchos \textit{pseudo-estándares} conviviendo en un mismo proyecto.
