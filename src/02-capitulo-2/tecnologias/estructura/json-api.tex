\paragraph{JSON API}
\label{soa:tecnologias:json-api}

JSON API es una especificación surgida en 2013 que intenta definir tanto cómo deben los clientes formular las peticiones, así como la forma en que los servidores deben responder a ellas, fomentando la eficiencia en el uso de recursos. Al momento de escritura del presente trabajo, esta especificación se encuentra en su versión \texttt{1.0} y con una versión \texttt{1.1} en proceso de definiciones.

En este análisis resumiremos de manera concisa los puntos sobresalientes de la especificación.

\subparagraph{\eng{Media type} dedicado}

Tanto los datos que el cliente envía en su petición como la respuesta del servidor deben indicar el \eng{media type} dedicado de JSON API: \texttt{application/vnd.api+json}\footnote{Este tipo de contenido fue asignado por la IANA - \url{http://www.iana.org/assignments/media-types/application/vnd.api+json}}.

Este tipo de contenidos es una definición de estructura basada en el lenguaje \gls{lang:json}. Denominaremos ``documentos JSON API'' a aquellos documentos \gls{lang:json} que adhieran a este \eng{media type}.

\subparagraph{Estructura general de los documentos JSON API}

Los documentos deben contener como elemento raiz un objeto, en el cual son admisibles las siguientes propiedades de primer nivel:

\begin{itemize}
  \item \texttt{data}: la información principal del documento, puede ser un recurso o una colección de éstos.
  \item \texttt{errors}: indicadores de cualquier error que hubiera ocurrido.
  \item \texttt{meta}: especifica metadatos sobre la información.
  \item \texttt{jsonapi}: descripción de la implementación del servidor JSON API.
  \item \texttt{links}: conjunto de vínculos hypermedia relacionados a la información principal.
  \item \texttt{included}: recursos incluidos en el documento por estar relacionados al objeto principal.
\end{itemize}

\subparagraph{Los recursos}

TODO
