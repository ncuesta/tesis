\paragraph{API Axle}

Este es otro producto que comenzamos a analizar ya que parece tener características similares a \nameref{soa:tecnologias:kong} y \nameref{soa:tecnologias:api-umbrella}:

\begin{itemize}
  \item Funciona como un proxy reverso que se ubica delante de los servicios que queremos que provea.
  \item Utiliza \nameref{soa:tecnologias:nginx} como base para atender requerimientos entrantes.
  \item Implementa limitación de tasa de consultas, autenticación por clave de acceso, cache de respuestas, entre otros.
  \item Almacena en una base de datos \gls{db:redis} las estadísticas de uso.
\end{itemize}

\subparagraph{Licencia}

Este producto está licenciado bajo GPL v3\footnote{\url{https://github.com/apiaxle/apiaxle/blob/develop/GPL-3.0.txt}}.

\subparagraph{Estado del proyecto}

Lamentablemente, al indagar un poco más en profundidad sobre este producto notamos que el proyecto no tiene actividad desde Mayo de 2015 y que su documentación es escasa e incompleta. La falta de ejemplos reales y de casos testigo que respalden el uso de la herramienta completan un panorama no muy alentador para considerar esta herramienta entre las opciones para nuestra nueva arquitectura.

\subparagraph{Conclusión}

La falta de mantenimiento, documentación y datos que respalden la posibilidad de usar API Axle seriamente en producción, sumados a la complejidad para instalar la herramienta para hacer una prueba de concepto nos detuvo al momento de continuar analizando este producto.
