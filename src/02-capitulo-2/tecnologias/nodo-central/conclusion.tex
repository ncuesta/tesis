\paragraph{Conclusión}

Kong o Tyk, porque...

\textit{TODO: Acomodar las conclusiones que estaban en los productos para unificarlas comparativamente:}

\subparagraph{Conclusión (tomada de API Axle)}

La falta de mantenimiento, documentación y datos que respalden la posibilidad de usar API Axle seriamente en producción, sumados a la complejidad para instalar la herramienta para hacer una prueba de concepto nos detuvo al momento de continuar analizando este producto.

\subparagraph{Conclusión (tomada de API Umbrella)}

API Umbrella se ve muy prometedor, es un producto más grande que otros que hemos analizado como \nameref{soa:tecnologias:kong}, lo cual da soporte para algunas características que no están presentes en el resto como el uso de \nameref{soa:tecnologias:varnish} como cache compartida dentro del mismo proxy, pero presenta un gran inconveniente para considerarlo como una opción en nuestra arquitectura: su nivel de madurez. En el diseño final para la nube de servicios necesitamos utilizar herramientas estables, listas para ser utilizadas en servicios en producción y API Umbrella aún no puede ser considerada lo suficientemente madura y probada como para ser una candidata fuerte para ser incluida en nuestra arquitectura.

\subparagraph{Conclusión (tomada de Kong)}

Kong es una excelente alternativa al uso de un \gls{acro:esb}, ya que es más liviano, sencillo de configurar y administrar, altamente extensible y personalizable, y cubre las necesidades que tenemos en cuanto a escalabilidad, transparencia, centralización de la gestión y portabilidad.
