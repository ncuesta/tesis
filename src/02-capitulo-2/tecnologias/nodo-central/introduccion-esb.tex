
Uno de los 9 principios de diseño de \gls{acro:soa} es el bajo acoplamiento (\eng{loose coupling}) de los servicios.  La forma más común de implementar bajo acoplamiento para los servicios \gls{acro:soa}, es mediante un \gls{acro:esb}.
\todo{ver si se trataron los 9 principios en SOA}
% http://simplicable.com/new/the-9-principles-of-soa-design

Pero que es un \gls{acro:esb}?

Un \gls{acro:esb} no implementa en sí mismo una arquitectura orientada a servicios (\gls{acro:soa}), sino que proporciona las características, mediante las cuales sí se puede implementarse.  Proporciona una capa abstracción para los \eng{endpoints}. De esta manera se consigue flexibilidad y una fácil conexión entre los servicios.

Existen diferentes opiniones acerca del rol exacto y de las responsabilidades de un \gls{acro:esb}.  Parte de esta razón, de diferentes comprensiones de un \gls{acro:esb}, es que hay diferentes aproximaciones técnicas para realizar un \gls{acro:esb} \cite[p.~47]{josuttis2007}.

En función de los enfoques técnicos y de organización adoptadas para la aplicación del \gls{acro:esb}, puede implicar las siguientes tareas:

\begin{itemize}
  \item Providing connectivity
  \item Data transformation
  \item (Intelligent) routing
  \item Dealing with security
  \item Dealing with reliability
  \item Service management
  \item Monitoring and logging
\end{itemize}

El rol principal de un \gls{acro:esb}, es proveer interoperabilidad.  Debido a que integra diferentes plataformas y lenguajes de programación, una parte fundamental de esta función es la transformación de datos.
Otra tarea fundamental de un \gls{acro:esb}, es el ruteo, debe existir alguna manera de acceder a un servicio desde un consumidor a un proveedor, y luego enviar la respuesta de vuelta desde el proveedor hasta el consumidor.
Dependiendo de la tecnología utilizada, y el nivel de inteligencia proporcionada, esta tarea puede ser trivial, o puede requerir procesamiento muy complicado.

Hay que tener en cuenta que no existe requerimiento alguno para que el \gls{acro:esb} sea homogéneo.  Aunque podría ser mejor usar una sola tecnología para la implementación de los servicios, raramente es el caso. \gls{acro:soa}, por su propia naturaleza, acepta heterogeneidad. Eso incluye la heterogeneidad en middleware y protocolos. Incluso con un estándar como web service, múltiples instancias pueden diferir.  Tarde o temprano, se introducirá un nuevo estándar o una nueva versión del estándar que hace las cosas mejor y más fácil. Tan pronto como empiece a utilizar el nuevo estándar (junto a la antigua tecnología), el \gls{acro:esb} se volverá heterogéneo.\cite[p.~49]{josuttis2007}

Idealmente, el cambio de tecnología en el \gls{acro:esb}, no debe tener ningún impacto en los proveedores y consumidores, deben ser capaces de utilizar la misma \gls{acro:api}, y sólo debe cabmiar el mapeo.
Es decir, desde el punto de vista de los proveedores y consumidores, la \gls{acro:api} de servicios debe ser transparente. Sin embargo, esto por lo general requiere que el \gls{acro:esb} incluya la \gls{acro:api} de servicios para cualquier plataforma específica. Si el \gls{acro:esb} requiere un sólo protocolo específico, los consumidores y proveedores tienen que lidiar con las modificaciones de este protocolo.\cite[p.~50]{josuttis2007}
