Siendo la parte de la arquitectura que mayor desarrollo y mantenimiento de nuestra parte implicará, decidimos utilizar el lenguaje Ruby para implementar los servicios web. Basándonos en nuestra experiencia utilizando Ruby para el desarrollo de aplicaciones web, herramientas y \eng{scripts}, seguimos eligiendo este lenguaje por los siguientes motivos:

\begin{itemize}
  \item \textbf{Robustez sin sacrificio de la elegancia:} con varias aplicaciones en producción desarrolladas en este lenguaje\footnote{La lista detallada de estas aplicaciones puede consultarse en el \nameref{anexo:detalle-clientes}}, hemos comprobado que es robusto y potente, pero también elegante y \textit{cómodo} de utilizar.
  \item \textbf{Agilidad:} implementar aplicaciones en Ruby, dada la gran oferta de librerías y frameworks que posee, es realmente ágil y práctico, principalmente si lo contrastamos con el desarrollo en lenguajes con procesos más largos de despliegue del código desde que se desarrolla hasta que se despliega en producción.
  \item \textbf{Sus bondades:} en estos más de 3 años realizando a diario desarrollos con Ruby, no hemos hecho más que apreciar cada vez más las bondades que tiene y los beneficios que nos trae, principalmente en comparación con experiencias anteriores utilizando otros lenguajes (principalmente PHP).
\end{itemize}

Complementariamente al lenguaje, hemos decidido utilizar un framework para desarrollo de aplicaciones web para sustentar nuestros servicios. El uso de un framework nos provee de una base probada, eficiente, estandarizada para nuestros proyectos, a la vez que nos asiste en cuestiones de seguridad fundamentales a la hora de exponer servicios en Internet. Es por eso que la elección del framework a usar no es un tema de menor importancia, ya que esperamos que éste nos asista y facilite las tareas básicas pertinentes al desarrollo de los servicios, permitiendo que el foco pueda mantenerse en la implementación de la lógica de los servicios y no verse desviada por la toma de decisiones triviales como, por citar un ejemplo, la organización del código del proyecto.

Partiendo de esa base, tomamos los dos frameworks para desarrollo de aplicaciones web más consolidados y utilizados\footnote{Para realizar esta apreciación nos basamos en las investigaciones que hemos hecho en el pasado como parte de nuestro trabajo en el {\cespi} y en los datos que el sitio \eng{The Ruby Toolbox} provee: \url{https://www.ruby-toolbox.com/categories/web_app_frameworks}.} en el ámbito de Ruby como posibles opciones con las cuales implementar nuestros servicios web: Ruby on Rails y Sinatra.

En este apartado describiremos ambos frameworks comparativamente para concluir en la elección que hemos hecho para el presente trabajo.
