\subsubsection{Apache}
\label{soa:tecnologias:apache}

Apache es un servidor web que surgió en 1995 y desde 1999 es desarrollado por la \eng{Apache Foundation}, fundación a la que dio origen. Desde su creación y durante muchos años, este producto fue el estándar \textit{de facto} a la hora de instalar servidores web, la opción \textit{indudable} cuando se buscaban tecnologías abiertas. Si bien en el presente existen otros grandes contendientes que se acercan lentamente para quitarle ese lugar, Apache ocupa el primer puesto entre los servidores web activos del mundo, según datos del sitio netcraft como puede verse en la \autoref{fig:netcraft-stats-web-servers}.

En su concepción más pura, Apache es un producto que funciona como servidor web, ya sea de sitios estáticos o dinámicos, lo cual lo dejaría fuera de nuestro análisis para este punto de nuestra arquitectura; pero esta funcionalidad básica puede ser extendida mediante la adición de \textit{módulos} que tienen propósitos específicos, como para nuestro caso, existe el módulo \verb|mod_proxy_balancer|\footnote{\url{http://httpd.apache.org/docs/2.2/mod/mod_proxy_balancer.html}}, el cual dota al servidor de la lógica necesaria para que funcione como balanceador de carga.

A la hora de considerar Apache como opción, debemos saber su modelo de diseño y las consecuencias que éste puede tener en nuestra arquitectura. El modelo de funcionamiento que posee este producto se basa en la creación de procesos e hilos de sistema operativo para manejar las conexiones entrantes, donde el número máximo de procesos que pueda crear se configura de antemano y es un factor determinante en la degradación de performance que puede sufrir el equipo donde se ejecuta este servidor, ya que si se crean demasiados procesos de Apache puede ocurrir que el equipo se quede sin memoria principal disponible y deba comenzar a realizar \eng{swapping} de la memoria al disco y viceversa. En el otro extremo, si se configura un límite bajo a esa cantidad de procesos disponibles, Apache rechazará las conexiones entrantes una vez que haya alcanzado ese número.

Esta limitación que presenta el modelo de Apache no es algo que podamos ignorar, ya que en momentos de alta carga puede llevar toda la arquitectura de la nueva nube de servicios a un colapso por un funcionamiento poco performante del nodo encargado de balancear la carga entrante.
