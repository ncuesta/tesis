Al planificar una arquitectura escalable, necesitamos tener una forma de agregar o quitar nodos de manera dinámica de forma que se logre elasticidad para ecalar horizontalmente. La forma más común de lograr esto es utilizando un nodo dedicado a la distribución del trabajo entrante entre aquellos capaces de atender la tarea entrante, lo cual no solo abstrae a las capas anteriores de la arquitectura sobre cuántos o qué nodos se encargan de realizar ese trabajo, si no que además permite distribuir más equitativamente esa carga entre ellos.

Esta forma de balanceo y escalabilidad horizontal es la que utilizaremos en nuestra propuesta, y a tal fin hemos considerado como opciones las dos herramientas de código abierto más populares y probadas de los últimos años. En los apartados presentados a continuación analizaremos brevemente cada una a fin de elegir una para utilizar en nuestra propuesta.
