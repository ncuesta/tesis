\subsubsection{NGINX}
\label{soa:tecnologias:nginx}

Este servidor web, que surgió en 2002 y se liberó en el 2004 como proyecto \eng{open source}, fue diseñado para solucionar las limitaciones que \nameref{soa:tecnologias:apache} tiene, principalmente en cuanto al excesivo uso de recursos y los problemas de escalabilidad que éstos acarrean.

La principal flexibilidad de este producto es resultado de su modelo de funcionamiento, el cual consiste en una arquitectura basada en eventos donde cada proceso \textit{worker} (como se denomina a los que atienden conexiones entrantes) puede atender miles de conexiones entrantes en simultáneo, ya que los workers sólo reaccionan ante eventos, en lugar de bloquearse durante la atención de una conexión entrante hasta que se termina de enviar la respuesta. Esto, en contraste con el modelo bloqueante que \nameref{soa:tecnologias:apache} posee es una gran ventaja, tanto en la reducción de consumos que provoca como en la eliminación de restricciones fácilmente alcanzables con respecto a la cantidad de conexiones entrantes admisibles.

NGINX nace como un producto orientado a los nuevos patrones arquitectónicos y de diseño de las aplicaciones web modernas, como la empresa detrás del producto, NGINX Inc., lo define al comparar NGINX con \nameref{soa:tecnologias:apache}\footnote{Cf. \url{https://www.nginx.com/blog/nginx-vs-apache-our-view}}. En esa publicación explican que, adicionalmente a poder ser productos complementarios, NGINX está diseñado para los nuevos modelos de aplicaciones haciendo especial hincapié en el modelo de microservicios, y para hacer las de proxy o balanceador de carga en una arquitectura que así lo requiera, soportando la mayor parte de la carga y distribuyéndola entre los \eng{backends} de procesamiento (los servicios, en nuestro caso) para que la atiendan.
