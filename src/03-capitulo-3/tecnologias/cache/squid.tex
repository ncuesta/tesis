\subsubsection{Squid}
\label{soa:tecnologias:squid}

\gls{db:squid} fue originalmente desarrollado para el proyecto \eng{Harvest} de la Universidad de Colorado Boulder.  Duane Wessels bifurcó el proyecto de la ``última versión pre-comercial de Harvest'', el cual fue renombrado a Squid para evitar confusiones con la versión comercial Cached 2.0, que se conviertió luego en NetCache.  La versión 1.0.0 de Squid fue liberada en Julio 1996.

\gls{db:squid} es un \eng{proxy server} (\eng{forward proxy server}) para la Web, soporta \gls{proto:http}, \gls{proto:https}, \gls{acro:ftp}, entre otros. Reduce el ancho de banda y mejora los tiempos de respuesta por el almacenamiento en caché y la reutilización de las páginas web frecuentemente solicitadas.  Puede implementarse en la mayoría de los sistemas operativos disponibles, incluyendo Windows y está disponible bajo licencia GNU GPL.

\gls{db:squid} optimiza el flujo de datos entre el cliente y el servidor, utilizando el contenido en caches frecuentemente usado, ahorrando de esta manera ancho de banda y obteniendo mejoras en el rendimiento de los clientes.

\paragraph{Instalación}

A continuación se detallan los pasos para instalar Squid en un servidor con Ubuntu 14.04.  Antes de comenzar con la instalación y configuración de Squid, debemos actualizar el software del servidor a su última versión, como se muestra en el bloque de código \autoref{soa:tecnologias:squid-cache:bash-preparacion}:

\begin{listing}[H]
  \bashfile{src/03-capitulo-3/tecnologias/cache/code/squid/00-preparacion.sh}
  \caption{Actualización del sistema de base}
  \label{soa:tecnologias:squid-cache:bash-preparacion}
\end{listing}

Una vez que hemos terminado la actulización del equipo, estamos en condiciones de iniciar la instalación de Squid.  Squid se encuentra disponible en los repositorios de Ubuntu, para instalarlo en el servidor debemos ejecutar el comando del bloque de código \autoref{soa:tecnologias:squid-cache01:bash-preparacion}:

\begin{listing}[H]
  \bashfile{src/03-capitulo-3/tecnologias/cache/code/squid/01-preparacion.sh}
  \caption{Instalación de Squid}
  \label{soa:tecnologias:squid-cache01:bash-preparacion}
\end{listing}

La configuración principal de Squid se encuentra en /etc/squid3/squid.conf.  Antes de realizar cualquier cambio en la configuración original, realizamos una copia del archivo con el comando del bloque de código \autoref{soa:tecnologias:squid-cache02:bash-preparacion}.

\begin{listing}[H]
  \bashfile{src/03-capitulo-3/tecnologias/cache/code/squid/02-preparacion.sh}
  \caption{Copia de respando de configuración de Squid}
  \label{soa:tecnologias:squid-cache02:bash-preparacion}
\end{listing}

A continuación procedemos a modificar el archivo principal de configuración de Squid con el editor \texttt{nano} (bloque de código \autoref{soa:tecnologias:squid-cache03:bash-preparacion}):

\begin{listing}[H]
  \bashfile{src/03-capitulo-3/tecnologias/cache/code/squid/03-preparacion.sh}
  \caption{Configuración de Squid}
  \label{soa:tecnologias:squid-cache03:bash-preparacion}
\end{listing}

Lo primero que debemos configurar es el puerto en el que Squid estará escuchando las peticiones, por defecto, Squid escucha las peticiones en el puerto \texttt{3128}.  Para cambiar el puerto por defecto, debemos editar la directiva \verb|http_port|.  En nuestro caso particular, configuramos el puerto \texttt{8888}, modificando la línea de esta directiva para que quede de la siguiente manera:\\ \verb|http_port 8888|

Por defecto, el servidor proxy HTTP no permite el acceso a nadie.  Para permitir el acceso al servidor desde cualquier IP, debemos editar la directiva \verb|http_access| para que quede de la siguiente manera:\\
\verb|http_access allow all|

Una vez que hemos realizado las configuraciones necesarias, debemos guardar los cambios y reiniciar el servicio de Squid, para que tome los cambios.  Para reiniciar el servio ejecutamos el comando del bloque de código \autoref{soa:tecnologias:squid-cache06:bash-preparacion}:

\begin{listing}[H]
  \bashfile{src/03-capitulo-3/tecnologias/cache/code/squid/06-preparacion.sh}
  \caption{Renicio del servicio Squid}
  \label{soa:tecnologias:squid-cache06:bash-preparacion}
\end{listing}

Para verificar el funcionamiento del servidor proxy, configuramos manualmente los datos de nuestro proxy en el navegador, ingresando la IP del servidor proxy y el puerto anteriomente configurado.
En caso de que tengamos problemas, podemos ver log \texttt{access.log} para obtener más información como se indica en el bloque de código \autoref{soa:tecnologias:squid-cache07:bash-preparacion}:

\begin{listing}[H]
  \bashfile{src/03-capitulo-3/tecnologias/cache/code/squid/07-preparacion.sh}
  \caption{Verificación del funcionamiento de Squid}
  \label{soa:tecnologias:squid-cache07:bash-preparacion}
\end{listing}
