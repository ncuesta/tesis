\subsubsection{Squid}
\label{soa:tecnologias:squid}

\gls{db:squid} es un proxy cache para la Web soporta \gls{proto:http}, \gls{proto:https}, \gls{acro:ftp}, entre otros. Reduce el ancho de banda y mejora los tiempos de respuesta por el almacenamiento en caché y la reutilización de las páginas web solicitadas frecuentemente.  Se ejecuta en la mayoría de los sistemas operativos disponibles, incluyendo Windows y está disponible bajo la GNU GPL.

\gls{db:squid} es usado por cientos de proveedores de internet en todo el mundo, para proporcionar a sus usuarios el mejor acceso posible a la web.  Optimiza el flujo de datos entre el cliente y el servidor, utilizando el contenido en caches frecuentemente usado, ahorrando de esta manera ancho de banda y obteniendo mejoras en el rendimiento de los clientes.

ver RFC 2186: Internet Cache Protocol (ICP), version 2
ver http://www.ietf.org/rfc/rfc2186.txt

Website Content Acceleration and Distribution
Thousands of web-sites around the Internet use Squid to drastically increase their content delivery. Squid can reduce your server load and improve delivery speeds to clients. Squid can also be used to deliver content from around the world - copying only the content being used, rather than inefficiently copying everything. Finally, Squid's advanced content routing configuration allows you to build content clusters to route and load balance requests via a variety of web servers.

\paragraph{Licencia}

\paragraph{Características principales}

\paragraph{Instalación y prueba}

\begin{listing}[H]
  \bashfile{src/03-capitulo-3/tecnologias/cache/code/squid/00-preparacion.sh}
  \caption{Instalación de Squid}
  \label{soa:tecnologias:squid-cache:bash-preparacion}
\end{listing}

\paragraph{Integración con nuestro diseño}
