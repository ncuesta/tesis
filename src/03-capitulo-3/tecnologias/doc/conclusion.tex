\subsubsection{Conclusión}

En esta sección hemos analizado herramientas para documentar \glspl{acro:api} \gls{acro:rest} que, en algunos casos, sirven para otros propósitos derivados del hecho de simplemente considerar la documentación de éstas como un texto descriptivo. Al tomar la documentación (o especificación) de un conjunto de servicios como un \textit{contrato}, ésta recibe una entidad mayor y gana importancia en el proceso de implementación de una \gls{acro:api}, ya que a partir de lo que en la documentación se defina se pueden generar tests automatizados que aseguren el correcto funcionamiento de los servicios, y a su vez, generen la necesidad de mantener actualizada esa especificación a medida que los servicios evolucionan.

Si bien las tres opciones consideradas comparten muchas bondades, hemos encontrado en \nameref{soa:tecnologias:openapi-spec} el modelo que más se ajusta a nuestras necesidades, y aquel que más se condice con nuestros principios de trabajo, al ser un estándar abierto impulsado por la Linux Foundation y mantenido por un consorcio compuesto de las más importantes empresas del sector.
