\subsubsection{Conclusión}

En nuestra búsqueda de formatos estándares para la estructura de la respuestas de nuestros servicios encontramos una gran variedad de opciones informales, propuestas desarrolladas por distintas empresas, pero sólo hallamos dos opciones que poseen la orientación adecuada para nuestras necesidades: \nameref{soa:tecnologias:hal} y \nameref{soa:tecnologias:json-api}.

El primer formato, \nameref{soa:tecnologias:hal}, es conciso y sencillo de implementar ya que no presenta grandes complicaciones en cuanto a sus requisitos funcionales. Si bien esto puede ser positivo, en esta situación nos resulta insuficiente para definir por completo el protocolo de acceso a nuestros servicios. En el mismo sentido, \nameref{soa:tecnologias:hal} no se encuentra en una versión definitiva, completa ni estable, lo cual lo elimina como candidato si consideramos que desde el año 2013 no recibe actualizaciones.

En el caso de \nameref{soa:tecnologias:json-api}, lo encontramos altamente beneficioso desde el punto de vista del desarrollo por su clara documentación, por tratarse de un estándar que define un protocolo de acceso a la información, por estar basado en las opiniones de varias empresas y equipos encabezados por arquitectos de renombre en la industria, por poseer una creciente adopción y por ser fácil de integrar con el framework que hemos elegido para implementar los servicios (\nameref{soa:tecnologias:rails}), y es por estos motivos que lo utilizaremos para estructurar las respuestas de nuestra nube de servicios.
