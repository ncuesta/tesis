La consistencia, coherencia y claridad en la estructura de las respuestas de los servicios que una nube provee es crítica para su usabilidad y mantenibilidad. En el diseño del Integrador hemos optado por definir nuestra propia estructura basada simplemente en el esquema de nuestras bases de datos: publicamos todos los campos indiscriminadamente, lo cual resulta en algunos casos en la exposición innecesaria y excesiva de valores en las respuestas de la \gls{acro:api}. Esto, sumado a una cierta inconsistencia en los nombres de los campos y en la forma en que se conforman las \glspl{acro:url} y sus parámetros, hacen difícil el uso de la nube sin consultas constantes a una documentación incompleta y desactualizada; consultas que en la myoría de los casos acaban por convertirse en una pérdida de tiempo y en la necesidad de leer el código fuente de las \glspl{acro:api} para conocer exactamente qué parámetros admite y qué efecto tienen en la petición.

En nuestro análisis de la nube actual éste fue uno de los puntos más importantes a mejorar: el requerimiento de tener una estructura estandarizada, fundada en casos de éxito, y que se integre con alguna herramienta de automatización de la documentación (veremos más sobre eso en la \autoref{soa:tecnologias:para-documentar}) para que podamos tenerla actualizada sin necesidad de hacerlo manualmente.

Con respecto al formato de la respuestas, hemos decidido mantener \gls{lang:json} como nuestra opción. Se trata de un formato muy amistoso para el desarrollador por su claridad y su relativamente poco \eng{overhead} para la codificación y decodificación, además de que se encuentra en uso en un creciente número de \glspl{acro:api} a nivel global\footnote{El interés en las \glspl{acro:api} \gls{lang:json} ha crecido constantemente desde el año 2004 al presente según las tendencias de Google. Cf. \url{http://www.google.com/trends/explore?q=xml\%20api\%2C\%20json\%20api\%2C\%20html\%20api}.}, por lo que nos centramos en la búsqueda de estándares específicos al formato \gls{lang:json}.

En este apartado analizaremos diferentes estándares, relacionados a la estructura de las respuestas y parámetros en las consultas, para el diseño y la implementación de una \gls{acro:api}.
