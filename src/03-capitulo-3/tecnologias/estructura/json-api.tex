\subsubsection{JSON API}
\label{soa:tecnologias:json-api}

JSON API es una especificación surgida en 2013 que intenta definir tanto cómo deben los clientes formular las peticiones, así como la forma en que los servidores deben responder a ellas, fomentando la eficiencia en el uso de recursos. Al momento de escritura del presente trabajo, esta especificación se encuentra en su versión \texttt{1.0} y con una versión \texttt{1.1} en proceso de definiciones.

En este análisis resumiremos de manera concisa los puntos sobresalientes de la especificación.

\paragraph{\eng{Media type} dedicado}

Tanto los datos que el cliente envía en su petición como la respuesta del servidor deben indicar el \eng{media type} dedicado de JSON API: \texttt{application/vnd.api+json}\footnote{Este tipo de contenido fue asignado por la IANA - \url{http://www.iana.org/assignments/media-types/application/vnd.api+json}}.

Este tipo de contenidos es una definición de estructura basada en el lenguaje \gls{lang:json}. Denominaremos ``documentos JSON API'' a aquellos documentos \gls{lang:json} que adhieran a este \eng{media type}.

\paragraph{Estructura general de los documentos JSON API}

Los documentos deben contener como elemento raiz un objeto, en el cual son admisibles las siguientes propiedades de primer nivel:

\begin{itemize}
  \item \texttt{data}: la información principal del documento, puede ser un recurso o una colección de éstos.
  \item \texttt{errors}: indicadores de cualquier error que hubiera ocurrido.
  \item \texttt{meta}: especifica metadatos sobre la información.
  \item \texttt{jsonapi}: descripción de la implementación del servidor JSON API.
  \item \texttt{links}: conjunto de vínculos hypermedia relacionados a la información principal.
  \item \texttt{included}: recursos incluidos en el documento por estar relacionados al objeto principal.
\end{itemize}

\paragraph{Los recursos}

Los elementos principales de información de un documento JSON API son los recursos. Cada recurso debe al menos tener dos propiedades: \texttt{id} y \texttt{type}, excepto cuando se trate de un \textit{nuevo} elemento que esté siendo creado por el cliente del servicio en cuyo caso el \texttt{id} puede omitirse. Además de estas propiedades obligatorias un recurso puede tener las propiedades \texttt{attributes}, \texttt{relationships}, \texttt{links} y \texttt{meta}, donde las dos primeras son objetos \gls{lang:json} que indican los atributos del recurso y posibles elementos relacionados respectivamente, y las últimas dos tienen sentidos similares a las propiedades homónimas del elemento raiz mencionadas en el apartado anterior que describe la estructura general de un documento JSON API. En el mismo sentido, los recursos pueden identificarse mediante una versión reducida de sí mismos que sólo contenga las propiedades \texttt{id} y \texttt{type}, caso en el cual se los denomina ``objetos identificadores de recursos''.

En este punto, la especificación hace una recomendación respecto de los campos (dentro de los \texttt{attributes} de un recurso) que representan \textit{asociaciones lógicas} o claves foráneas a otros recursos: los valores de clave foránea apuntando hacia otros recursos no deberían aparecer entre los atributos del recurso; en cambio debieran tener su entrada dedicada entre las \textit{relationships} del mismo.

Las relaciones de un recurso son objetos \glspl{lang:json} que pueden tener las siguientes propiedades:

\begin{itemize}
  \item \texttt{links}: un objeto \glspl{lang:json} que puede tener dos propiedades, \texttt{self} (\gls{acro:url} mediante la cual se puede manipular la relación en sí) y \texttt{related} (\gls{acro:url} del recurso asociado en sí, que no debe ser dependiente de la existencia de la relación).
  \item \texttt{data}: uno o varios (acorde a la cardinalidad de la relación) objetos identificadores de recursos para los recursos asociados.
  \item \texttt{meta}: un objeto \glspl{lang:json} con metadatos sobre el recurso.
\end{itemize}

La especificación, con miras a hacer un uso más eficiente de los recursos, define también los ``documentos compuestos'' que son aquellos que incluyen en la misma estructura la información relacionada a los recursos asociados a la información principal del documento, es decir, aquellos recursos referenciados en alguna de las relaciones. La información de estos recursos adicionales debe realizarse dentro de una clave \texttt{included} de primer nivel en la raiz del documento JSON API. Al incorporar esta información adicional al documento, se reducen la cantidad de peticiones necesarias para obtener un recurso y sus elementos relacionados, reduciendo la utilización de recursos necesaria para completar la tarea.

\begingroup
  \jsonfile{src/03-capitulo-3/tecnologias/estructura/code/json-api/recurso.json}
  \captionof{listing}{Documento JSON API representando un recurso\label{soa:tecnologias:json-api:recurso}}
\endgroup

\paragraph{Obtención de recursos}

Las peticiones de acceso de lectura a la información deben ser realizadas utilizando el método \gls{proto:http} \texttt{GET} y utilizando alguna de las \glspl{acro:url} provistas en los documentos JSON API como vínculos \texttt{self} o \texttt{related}. Por ejemplo, las peticiones presentadas a continuación son válidas para obtener información a partir del documento JSON API presentado anteriormente.

Esta petición se puede utilizar para obtener la colección de tipos de documentos:

\begin{listing}[H]
  \httpfile{src/03-capitulo-3/tecnologias/estructura/code/json-api/obtener-coleccion.http}
  \caption{Petición de una colección de recursos JSON API}
  \label{soa:tecnologias:json-api:obtener-colecccion}
\end{listing}

Esta otra para obtener todas las personas relacionadas con el tipo de documento con \texttt{id} \texttt{"0"}:

\begin{listing}[H]
  \httpfile{src/03-capitulo-3/tecnologias/estructura/code/json-api/obtener-recurso-relacionado.http}
  \caption{Petición de un recurso relacionado en JSON API}
  \label{soa:tecnologias:json-api:obtener-recurso-relacionado}
\end{listing}

Y la siguiente para obtener una persona en particular:

\begin{listing}[H]
  \httpfile{src/03-capitulo-3/tecnologias/estructura/code/json-api/obtener-recurso.http}
  \caption{Petición de un recurso en JSON API}
  \label{soa:tecnologias:json-api:obtener-recurso}
\end{listing}

Las respuestas exitosas a estas peticiones deben devolver un código de estado \gls{proto:http} \texttt{200 OK} junto con el recurso (o un elemento nulo, en caso de no existir el recurso identificado y estar siendo solicitado a través de una relación) o la colección de recursos (o una colección vacía, en caso de no poseer elementos). En el caso de las peticiones para obtener un único recurso inexistente que no sean realizadas a través de la \gls{acro:url} de una relación, el servidor deberá responder con un código de estado \gls{proto:http} \texttt{404 Not Found}.

El documento JSON API expuesto en el bloque de código \autoref{soa:tecnologias:json-api:recurso} es un ejemplo del cuerpo de una respuesta exitosa que debiera responder con el siguiente encabezado \gls{proto:http}:

\begin{listing}[H]
  \httpfile{src/03-capitulo-3/tecnologias/estructura/code/json-api/respuesta-200-ok.http}
  \caption{Encabezado HTTP de respuesta exitosa JSON API}
  \label{soa:tecnologias:json-api:respuesta-200-ok}
\end{listing}

Similarmente, el extracto presentado en el bloque de código \autoref{soa:tecnologias:json-api:respuesta-200-obtener-coleccion} representa una respuesta exitosa a una petición sobre la colección de tipos de documento.

\begin{listing}
  \httpfile{src/03-capitulo-3/tecnologias/estructura/code/json-api/respuesta-200-obtener-coleccion.http}
  \caption{Respuesta exitosa a petición de una colección de recursos en JSON API}
  \label{soa:tecnologias:json-api:respuesta-200-obtener-coleccion}
\end{listing}

Si en cambio la petición fuese exitosa pero su respuesta no contuviese elementos, las respuestas serían las presentadas en los bloques de código \autoref{soa:tecnologias:json-api:respuesta-200-recurso-nulo} (para un recurso singular) y \autoref{soa:tecnologias:json-api:respuesta-200-coleccion-vacia} para una colección.

\begin{listing}
  \httpfile{src/03-capitulo-3/tecnologias/estructura/code/json-api/respuesta-200-recurso-nulo.http}
  \caption{Respuesta JSON API para una petición exitosa a un recurso no encontrado}
  \label{soa:tecnologias:json-api:respuesta-200-recurso-nulo}
\end{listing}

\begin{listing}
  \httpfile{src/03-capitulo-3/tecnologias/estructura/code/json-api/respuesta-200-coleccion-vacia.http}
  \caption{Respuesta JSON API para una petición exitosa a una colección de recursos vacía}
  \label{soa:tecnologias:json-api:respuesta-200-coleccion-vacia}
\end{listing}

En los ejemplos anteriores se demuestra el uso del valor nulo (\texttt{null}) y la colección vacía (\texttt{[]}) para denotar la respuesta a una petición exitosa pero que no arroja coincidencias.

\clearpage
\paragraph{Obtención de relaciones}

Como se mencionó anteriormente, las relaciones pueden tener un vínculo a sí mismas (\texttt{self}) que permita manipular la relación en sí y no a los recursos que ésta vincula. Esa \gls{acro:url} consultada con el método \gls{proto:http} \texttt{GET} nos permite consultar la información asociada a esa relación, obteniendo como dato principal el recurso o los recursos involucrados en ella en forma de identificadores de recursos (objetos \gls{lang:json} con las propiedades \texttt{id} y \texttt{type} únicamente).

El servidor responderá con un código de estado \gls{proto:http} \texttt{200 OK} en caso de tratarse de una consulta exitosa, y con un \texttt{404 Not Found} si se tratase de una petición a una relación sobre un recurso principal inexistente. A modo de ejemplo, se presentan estos dos posibles casos en situaciones concretas:

\begin{itemize}
  \item \verb|GET /people/000000000000000000000000027855859/relationships/document_type|\\
    Esta petición, realizada sobre el recurso principal de tipo \texttt{people} con identificador \texttt{000000000000000000000000027855859}, pide el recurso de tipo \verb|document_type| relacionado. Considerando que el recurso principal existe, se retornará invariablemente un código de estado \gls{proto:http} \texttt{200 OK} y una respuesta cuyo atributo \texttt{data} contendrá o bien un recurso existente, o bien un valor nulo (\texttt{null}) en caso que dicho recurso no exista.
  \item \verb|GET /people/inexistente/relationships/document_type|\\
    Esta petición, similar a la anterior pero realizada sobre un ficticio recurso principal inexistente, tendrá una respuesta con estado \gls{proto:http} \texttt{404 Not Found} indicando que es el recurso principal el que no existe, y por ende descartando la posibilidad de que la relación solicitada pueda ser obtenida.
\end{itemize}

\begin{listing}
  \httpfile{src/03-capitulo-3/tecnologias/estructura/code/json-api/respuesta-200-obtener-relacion.http}
  \caption{Respuesta JSON API para una petición exitosa a una relación}
  \label{soa:tecnologias:json-api:respuesta-200-obtener-relacion}
\end{listing}

\paragraph{Inclusión de recursos relacionados}

El estándar define una forma para reducir la cantidad de peticiones realizadas a los servicios mediante la inclusión de recursos secundarios o \textit{relacionados} al recurso o conjunto de recursos principal. La forma de indicar que se desea utilizar esta funcionalidad es mediante la especificación del parámetro \texttt{include} con un valor que contenga una o más relaciones del recurso (o recursos asociados) separadas por coma. Por ejemplo, el valor presentado en el bloque \autoref{soa:tecnologias:json-api:obtener-recurso-include} indicaría que se incluyan recursos relacionados.

\begin{listing}
  \httpfile{src/03-capitulo-3/tecnologias/estructura/code/json-api/obtener-recurso-include.http}
  \caption{Petición de un recurso y relaciones asociadas en JSON API}
  \label{soa:tecnologias:json-api:obtener-recurso-include}
\end{listing}

\paragraph{Selección de campos}

Otra forma incluida en la especificación orientada a la reducción del uso de recursos son los \eng{sparse fieldsets}, recursos que son consultados indicando específicamente qué campos se deben incluir en la respuesta. Así como el apartado anterior explica una forma de reducir requerimientos, este explica una forma de reducir el tamaño de cada respuesta, e inclusive ambos pueden combinarse para reducir la cantidad de campos de diferentes recursos en una misma respuesta JSON API. Es justamente por este tipo de situaciones en que una misma respuesta podría incluir diferentes recursos de diferentes tipos que la forma de especificar qué campos se desean obtener en la respuesta es mediante el parámetro \texttt{fields} el cual es un arreglo de claves y valores, donde las claves son tipos de recursos y los valores los campos a incluir en la respuesta. A continuación, en el bloque de código \autoref{soa:tecnologias:json-api:obtener-recurso-fields}, se presenta un ejemplo de petición utilizando esta posibilidad definida por JSON API.

\begin{listing}
  \httpfile{src/03-capitulo-3/tecnologias/estructura/code/json-api/obtener-recurso-fields.http}
  \caption{Petición de un recurso utilizando \eng{sparse fieldsets} e \texttt{include} en JSON API}
  \label{soa:tecnologias:json-api:obtener-recurso-fields}
\end{listing}

\paragraph{Especificación de orden}

Al realizar una petición de una colección, donde normalmente se retornará más de un elemento, la especificación reserva el parámetro \texttt{sort} para indicar diferentes criterios de ordenamiento de los recursos en la respuesta. El formato para el valor del parámetro \texttt{sort} es una lista de campos, separados por coma, en la que se puede anteponer un símbolo menos (\texttt{-}) a cada campo en que se quiera indicar un orden descendente. En el bloque de código \autoref{soa:tecnologias:json-api:obtener-recurso-sort} se puede observar un ejemplo de esto.

\begin{listing}
  \httpfile{src/03-capitulo-3/tecnologias/estructura/code/json-api/obtener-recurso-sort.http}
  \caption{Petición indicando el orden de una colección de recursos en JSON API}
  \label{soa:tecnologias:json-api:obtener-recurso-sort}
\end{listing}

\paragraph{Paginación de colecciones}

JSON API define también una forma estandarizada de limitación de la cantidad de recursos incluidos en cada respuesta a las peticiones realizadas a las colecciones, la cual llama \textit{paginación}. Esta técnica divide en \textit{páginas} de una cierta cantidad de elementos (\textit{tamaño de página}), accesibles mediante un índice o \textit{número de página}, siendo \texttt{1} el número de la primera página.

A fin de que los clientes puedan especificar qué tamaño de página desean utilizar y qué número de página desean obtener, JSON API reserva el parámetro \texttt{page} y define que será un arreglo de clave-valor con dos claves posibles: \texttt{size} y \texttt{number}, para indicar el tamaño y el número de página, respectivamente. En el bloque de código \autoref{soa:tecnologias:json-api:obtener-recurso-page} se muestra un ejemplo de una petición que indica estos parámetros. Si el cliente no especificase nada respecto de la paginación, el servidor puede implementar un comportamiento por defecto a su criterio.

\begin{listing}
  \httpfile{src/03-capitulo-3/tecnologias/estructura/code/json-api/obtener-recurso-page.http}
  \caption{Petición indicando parámetros de paginación en JSON API}
  \label{soa:tecnologias:json-api:obtener-recurso-page}
\end{listing}

\paragraph{Filtrado de resultados}

Adicionalmente a las posibilidades antes mencionadas, JSON API permite que se filtren los resultados de una petición mediante la palabra clave \texttt{filter}, cuyos valores serán los criterios de búsqueda que el cliente envíe al servidor para reducir el número de recursos únicamente a los que apliquen a dichos criterios. La forma concreta de interpretación de este parámetro se deja abierta a la implementación del servidor.
