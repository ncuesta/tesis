\subsubsection{Conclusión}

En esta sección hemos analizado variantes para abordar un mismo problema, la centralización de la distribución de conexiones entre los clientes y los servicios sin generar nuevos cuellos de botella. En el análisis hemos considerado desde un \gls{acro:esb} completo, con toda su estructura, hasta tecnologías que simplemente funcionaban de proxy reverso altamente performante con la posibilidad de administrarlo y algún agregado más.

Este amplio espectro de opciones nos ayudó a comprender mejor nuestras necesidades y, por ende, a tomar una decisión basada en nuestras necesidades y no simplemente en las tendencias de la industria. En un extremo del espectro de opciones nos encontramos con un \gls{acro:esb} completo, lo cual nos resulta demasiado para el planteo que estamos haciendo, ya que incluirlo agregaría capas innecesarias de complejidad a nuestro nodo central. En el opuesto nos encontramos con productos en extremo simples (o tal vez incompletos) que nos generan la necesidad de implementar nosotros algunos elementos para nuestra arquitectura, como ser analíticas o monitoreo de \eng{endpoints}.

En un punto relativamente intermedio del rango encontramos opciones como \nameref{soa:tecnologias:api-umbrella} y \nameref{soa:tecnologias:tyk}, que nos brindan un buen equilibrio entre prestaciones y complejidad, sin sacrificar nuestras necesidades para este nodo central. Entre estos productos, el aspecto determinante para nuestra elección se basa en su madurez: al momento de realizar este trabajo, \nameref{soa:tecnologias:api-umbrella} no tiene un nivel de estabilidad adecuado para nuestras necesidad, mientras que \nameref{soa:tecnologias:tyk} sí, por lo cual elegimos este último para implementar el nodo central.
