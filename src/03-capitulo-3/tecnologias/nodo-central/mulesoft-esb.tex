\subsubsection{Mulesoft ESB}
\label{soa:tecnologias:mulesoft-esb}

Mule ESB es un \gls{acro:esb} liviano basado en Java, una plataforma de integración que permite a los desarrolladores conectar aplicaciones rápido y fácilmente.  Posibilita una integración sencilla de los sistemas existentes, independientemente de las tecnologías utilizadas para las aplicaciones, incluidas JMS, Web Services, JDBC y HTTP, entre otras.

La principal ventaja de un ESB es que permite que diferentes aplicaciones se comuniquen entre sí, actuando como un sistema de transporte para transmitir datos entre las aplicaciones dentro de la organización o a través de Internet. Mule ESB posee potentes capacidades, entre ellas:

\begin{itemize}
  \item Creación y alojamiento de servicios.
  \item Mediación de servicios.
  \item Ruteo de mensajes.
  \item Transformación de datos.
\end{itemize}

\paragraph{Licencia}

Licencia CPAL para \eng{Community Edition}, propiedad de Enterprise Edition.

\paragraph{Aplicabilidad del proyecto}

Mule ESB es un proyecto que no permite su libre distribución debido a su esquema de licenciamiento. Inicialmente incluimos este producto en nuestro análisis por poseer cierto renombre en la materia, pero al ahondar en sus detalles y conocer su modelo de negocios hemos decidido descartarlo automáticamente, al no ajustarse a nuestro propósito de utilizar tecnologías \eng{Open Source}.
