\subsection{Caso testigo}
\label{caso-testigo}

Habiendo desarrollado la propuesta de rediseño que motiva el presente trabajo, creemos conveniente tomar un caso testigo que sirva de ejemplo práctico para hacer concretos los conceptos teóricos que hemos tenido en cuenta para el análisis, a la vez que sirva de disparador para una implementación preliminar acotada del nuevo diseño sobre los elementos de la nube de servicios. De aquí en más, para distinguir la nube actual (el Integrador) de la nueva arquitectura que estamos planteando, utilizaremos el nombre clave \cloud para hacer referencia a esta última.

\subsubsection{Alcance}

La situación que utilizaremos como caso testigo es el proceso de registro de un nuevo usuario de \gls{acro:sso} de la nube de servicios de la \unlp. Si bien \textit{a priori} puede parecer un tanto sencillo para tomarlo de referencia, una vez que definamos en concreto todos los pasos involucrados en este proceso se hará evidente la elección realizada.

Aquellos agentes que forman parte del personal de la UNLP pueden tener un usuario único para acceder a las aplicaciones que nuestra Dirección desarrolla, ya sea para consultar sus recibos de haberes, utilizar una aplicación como parte de sus tareas diarias, presentar proyectos de extensión cuando la convocatoria se encuentre abierta o para acceder a cualquier aplicación que a futuro pudiéramos publicar. Para registrar su usuario, el agente dispone de dos vías:

\begin{itemize}
  \item La vía analógica: solicitando la creación del usuario, de manera presencial y por escrito, a la oficina de Personal de alguna de las Dependencias donde desempeña sus funciones. Por tratarse de un proceso manual y \eng{offline}, no lo consideraremos para este caso testigo.

  \item La vía digital: utilizando la \hyperref[anexo:detalle-clientes:sso]{aplicación de autogestión} que hemos desarrollado para realizar el registro del nuevo usuario en línea. \textit{Esta} es la vía que utilizaremos en nuestro planteo.
\end{itemize}

En el proceso de autogestión de un nuevo usuario, el agente debe completar los datos solicitados en una serie de pasos preestablecidos que lo guían hasta llegar a obtener el acceso a las aplicaciones que utilizan este esquema de \gls{acro:sso}\footnote{En realidad, inicialmente obtiene acceso únicamente a ver sus \hyperref[anexo:detalle-clientes:recibos]{recibos de haberes} y cargar su \textit{currículum de extensionista} en la aplicación de \hyperref[anexo:detalle-clientes:extension]{Proyectos de Extensión}. Para ingresar al resto de las aplicaciones un usuario autorizado le debe asignar los roles necesarios para cada aplicación.}. Podemos resumir estos pasos para el registro en:

\begin{enumerate}
  \item El agente ingresa a la aplicación de autogestión e indica que desea registrar un nuevo usuario. Para esto, debe ingresar una dirección de correo institucional propia para recibir por esa vía un vínculo de acceso para comenzar efectivamente el registro de su nuevo usuario.

  \item Al ingresar al vínculo de acceso que le llega a su correo, la aplicación solicita al agente que se identifique mediante su tipo y número de documento de identidad y en respuesta a ésto le indica si se lo tiene entre los agentes de la Universidad que aún no tienen usuario de acceso único.

  \item Una vez identificado el agente, se le sugiere un nombre de usuario siguiendo el formato estándar de nombres de usuario, el cual puede ser modificado como parte de la carga de información que se está realizando. Se chequea que este usuario no se encuentre en uso y que respete dicho formato estándar.

  \item Luego de elegido el nombre de usuario, se permite al agente ingresar una dirección de correo electrónico alternativa para tener un segundo medio de contacto.

  \item Para cerrar la carga de datos personales, se le pide que indique en qué Dependencias de la UNLP presta servicios. Esto se hace a modo de \gls{term:captcha} para evitar \eng{bots} que intenten registrar usuarios en masa y personas malintencionadas que deseen registrar usuarios en nombre de agentes reales.

  \item Al validar correctamente esta información, se le presentan todos los datos ingresados para su confirmación y si el agente los acepta, se le envía un correo electrónico con un formulario para presentar firmado en la oficina de Personal de alguna de las Dependencias donde presta servicios, teniendo un paso de validación humana de la información\footnote{Este paso adicional, si bien puede sonar contradictorio con el resto del proceso \eng{online} planteado, fue requerido por las autoridades de la Universidad para fortalecer los chequeos realizados sobre los datos recibidos mediante este servicio público de registro.}.

  \item Luego de presentar ese formulario y que sus datos sean debidamente corroborados por algún empleado de la oficina de Personal correspondiente, el usuario es aprobado y se dispara el envío de un nuevo correo al agente, en el cual se le consignan su usuario y una clave provisoria para que ingrese por primera vez a los sistemas de la UNLP. Con este paso, se finaliza el proceso de registro del nuevo usuario para el agente.
\end{enumerate}

Desde el punto de vista de nuestra aplicación, este proceso se ve un tanto reducido ya que no es necesario que incluyamos los pasos que ocurren por fuera del \textit{mundo virtual} o en la aplicación de gestión que utilizan los empleados de las oficinas de Personal de las Dependencias, ya que quedan afuera de su alcance. Formalizaremos los pasos que se realizan en este sistema con el diagrama de flujos presentado a continuación para su mayor referencia.

\begin{figure}[H]
  \centering
  \includegraphics[width=\textwidth,height=0.5\textheight,keepaspectratio]{src/images/05-capitulo-5/diagrama-flujo-registro.jpg}
  \caption{Proceso de autoregistro de un nuevo usuario de acceso único}
  \label{fig:diagrama-flujo-registro}
\end{figure}

Realizando una descomposición en servicios del proceso anterior, identificamos las siguientes dependencias con servicios de la \cloud:

\begin{itemize}
  \item \textbf{Servicios de referencia:} para visualizar y marcar las Dependencias (o Unidades Académicas) y para ofrecer para su selección los tipos de documento de identidad.

  \item \textbf{Servicios de información sobre el personal:} como vía para identificar a la persona y luego consultar las Unidades Académicas en las cuales presta sus servicios.

  \item \textbf{Servicios de usuarios:} para la consulta de la existencia de un usuario para la persona seleccionada, sugerir nombres de usuario que respeten el formato estándar definido y para la creación del nuevo usuario.

  \item \textbf{Servicio de notificaciones:} para el envío de correos electrónicos. Destacamos en este punto que para limitar el alcance del caso testigo dejaremos la implementación de este servicio como un trabajo a futuro, ya que su existencia no es limitante para poder desarrollar los pasos de registro que se realizan en línea.
\end{itemize}

El desarrollo del prototipo funcional implica:

\begin{itemize}
  \item Implementar desde cero los servicios de la \cloud antes mencionados (de referencia, de información sobre el personal y de usuarios), en los cuales incluiremos únicamente los \eng{endpoints} necesarios para solventar la lógica del caso testigo.

  \item Implementar desde cero una nueva versión de la aplicación cliente de registro, respetando los pasos antes descriptos, que consuma toda la información de los servicios de la \cloud. Esta nueva versión difiere en su concepción de la versión actualmente implementada en que no tendrá acceso a la base de datos de usuarios. De hecho, no tendrá acceso a \textit{ninguna} base de datos, ya que realizará todas sus operaciones de manera volátil dependiendo completamente de los servicios de la \cloud para acceder a la capa de persistencia en base de datos.

  \item Desarrollar una \textit{gema} (nombre que reciben las librerías reutilizables en el lenguaje Ruby) que encapsule la lógica de acceso a los servicios, desde la autenticación hasta la consulta y abstracción en objetos de las respuestas de sus \glspl{acro:api}, basándose en el estándar elegido para la codificación de la información.
\end{itemize}

Hemos elegido este caso testigo porque se compone de una serie de interacciones entre distintos servicios que brindan una noción de la complejidad que pueden tener las operaciones a realizar utilizando la nube de servicios, más allá de las simples consultas de datos de referencia que habitualmente manejamos. El hecho de incorporar diferentes servicios, englobados en distintas áreas de acción, nos permite también mostrar cómo funcionan las instancias de las aplicaciones que proveen esos servicios, cómo interactúan con la aplicación cliente y con los elementos intermediarios de la comunicación (entiéndase \eng{caches} compartidas, balanceadores de carga, \eng{proxies} reversos y la capa de mediación ofrecida por \nameref{soa:tecnologias:kong}).

En este informe intentaremos no entrar en detalles innecesarios sobre la implementación de la aplicación cliente ni de la capa de lógica del negocio de los servicios, para sí ahondar sobre temas relevantes desde el punto de vista de la comprensión de las entidades participantes en las comunicaciones. De todas formas, se deja disponible el código fuente de los distintos apartados para su referencia en \todo{Agregar link al código fuente de la implementación}.


\subsubsection{Desarrollo del prototipo funcional}

\todo{Documentar el desarrollo}
