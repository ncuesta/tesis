\subsection{Trabajos a futuro}
\label{trabajos-a-futuro}

A partir del presente informe y las experiencias obtenidas, tanto a partir de la formalización teórica de los conceptos involucrados en el diseño de una arquitectura orientada a servicios, como al llevarlos a la práctica en un caso concreto, consideramos que los puntos presentados a continuación permitirían extender lo aquí desarrollado.

\subsubsection{Automatización de la documentación}

Si bien hemos analizado productos para documentar y mantener la documentación de las \glspl{acro:api} desarrolladas, notamos que sería altamente beneficioso definir un proceso automático de actualización y publicación de la documentación de las mismas que se inicie cada vez que se \eng{pushee} una nueva versión del código de los servicios.

Esto mantendría siempre actualizada la documentación de los servicios, que deberían estar publicadas en un portal para desarrolladores que sirva para los integrantes de nuestra Dirección y para cualquier otro interesado en consumir la información pública de la {\cloud} de la {\unlp}.

\subsubsection{Automatización de la arquitectura}

En nuestras pruebas, hemos manejado de manera manual la instalación y configuración de la arquitectura base que utilizamos. Si bien esto puede funcionar bien para un caso testigo como el que aquí hemos presentado, al momento de llevar una arquitectura completa a producción es conveniente tener una forma automatizada, documentada y replicable de poner en funcionamiento todos sus nodos, como pueden ser \eng{scripts} de provisionamiento de los servidores involucrados.

En la actualidad, existe una cantidad considerable de herramientas que asisten en esta tarea, como por ejemplo Chef\footnote{\url{https://www.chef.io}}, Ansible\footnote{\url{https://www.ansible.com}} o Puppet\footnote{\url{https://puppetlabs.com}}, por nombrar algunos.

Se deja planteada para el futuro la necesidad de implementar esto con alguna herramienta afin, sea alguna de las aquí mencionadas u otra.

\subsubsection{Extensión de la gema cliente desarrollada}

El desarrollo que realizamos para dar soporte a las necesidades de la {\cloud} en nuestro caso testigo está limitado a aquellos puntos que se necesitó implementar. Aunque esto es un buen punto de partida y la gema resultante es completamente funcional, su alcance no sería suficiente para la implementación completa de la nueva nube de servicios.

Queda para etapas posteriores al presente trabajo la tarea de completar la lógica presente en la gema, agregando soporte para los nuevos \glspl{term:endpoint} que pudieran surgir, así como funcionalidad que asista al desarrollo de las aplicaciones cliente que no hayan sido necesarias para nuestro caso testigo.

\subsubsection{Implementación de pruebas}

Una buena práctica a la hora de desarrollar software es mantener una batería de pruebas a realizar sobre el producto para cerciorarnos que su ejecución genera los resultados esperados, y para garantizar que durante la evolución y el mentenimiento del mismo no se modifiquen su lógica de manera que impacte negativamente en dichos resultados. En el desarrollo de nuestro caso testigo obviamos la implementación de ese tipo de pruebas, dejando esto como una deuda técnica a saldar cuando este diseño comenzase a pasarse a desarrollo real. Recomendamos la realización de pruebas de unidad dentro de cada \gls{acro:api}, y de pruebas de integración entre las \glspl{acro:api} y las aplicaciones cliente, para garantizar que ningún cambio realizado en cualquiera de las partes intervinentes en la arquitectura produzca efectos secundarios no deseados en el resto.
