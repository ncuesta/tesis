\subsection{Conclusión}
\label{conclusion}

Comenzamos esta tesina a partir de la existencia de una problemática que necesitaba ser abordada desde distintos aspectos: el teórico, el cual nos daría bases firmes sobre las cuales plantear un nuevo diseño de la arquitectura de la nube, y el práctico, en el cual aplicaríamos los nuevos conocimientos adquiridos y adoptaríamos técnicas y tecnologías modernas para plantear un caso testigo.

Durante el proceso de investigación del estado del arte en la materia, nuestro enfoque inicial fue modificándose a partir de la asimilación de nuevos conceptos. El mayor cambio que experimentó nuestra concepción de la nueva arquitectura fue al ahondar en los microservicios (\autoref{microservicios}), patrón que nos resultó natural y adecuado para la {\cloud} y su nuevo diseño. De manera similar, al conocer productos alternativos a los \glspl{acro:esb} (\autoref{esb:introduccion} y \autoref{soa:tecnologias:para-nodo-central}) que preliminarmente habíamos relevado vimos la posibilidad de usar productos específicamente diseñados para la gestión de \glspl{acro:api} de servicios, delegando en éstos tareas como la autenticación, limitación de uso de recursos, registro de eventos y accesos, y la administración de clientes y medios para su acceso a la información.

Al hacer la prueba de concepto, obtuvimos una noción palpable del costo aparejado a implementar los servicios siguiendo el patrón de microservicios y a llevar estos cambios a nuestras aplicaciones. Ese experimento le da un alto valor agregado a este trabajo, ya que al momento de estimar el tiempo y los recursos necesarios para plasmar este cambio en los desarrollos existentes y futuros de nuestro equipo, podremos referirnos a esta experiencia. Ése es un punto clave dentro del camino que nos falta transitar para llegar al ambiente de producción con la {\cloud}, proceso que deberemos abordar, ya fuera del marco del presente trabajo, con el resto de nuestro equipo para planificar la transición ordenada de las aplicaciones existentes a la nueva arquitectura.

Con la implementación de una porción de la nueva arquitectura también pudimos experimentar los beneficios del uso del \hyperref[microservicios]{patrón de microservicios}\cite[p.~27]{richards2015}. Los puntos más fuertes derivados del uso de este patrón fueron la agilidad obtenida para el desarrollo, la facilidad para realizar despliegues independientes de los servicios en el ambiente de producción y, por sobre todas las cosas, el alto grado de desacoplamiento que acaban por tener los servicios entre sí y con respecto a las aplicaciones que los utilizan. Al separar los servicios por alcance, no presenta dificultades respetar el principio de una sola responsabilidad; y al separar las aplicaciones de los servicios, centralizando la lógica de negocios en estos últimos, se simplifican las aplicaciones clientes y se eliminan los posibles puntos de duplicación de código que pudieran existir en caso de tener la misma lógica de negocios presente en más de una aplicación\footnote{Este tipo de situaciones no nos es ajena, ya que en la actualidad tenemos diferentes aplicaciones que tienen en común parte de su lógica de negocios y por eso acaban duplicando porciones de código, con el impacto negativo que esa repetición conlleva.}. Asimismo, la arquitectura se puede extender e incluir nuevos \eng{service components} de manera transparente y sencilla, también debido al uso de este patrón y a la disposición de un \eng{broker} que funciona de intermediario.

La arquitectura diseñada logra mejoras más allá de las que nos propusimos inicialmente, producto de la investigación y aplicación de conceptos que al momento de comenzar esta tesina desconocíamos, cuanto menos, formalmente.

Como hemos expuesto en el apartado \textit{\nameref{caso-testigo:experiencia}} del caso testigo (\autoref{caso-testigo:experiencia}), consideramos que la elección de tecnologías para llevar a la práctica el diseño teórico fue beneficiosa y acertada, y que el producto final palpable que hemos generado es un punto de partida concreto para transmitir lo aprendido en este trabajo.

Este trabajo nos deja una propuesta para la arquitectura de la nueva {\cloud} de la {\unlp} que posee las características presentadas a continuación.

\begin{itemize}
  \item \textbf{Redundante:} admite la inclusión de nodos replicados en los distintos roles sin necesidad de modificar su diseño \textit{lógico}.
  \item \textbf{Extensible:} permite de manera transparente y sencilla incluir nuevos servicios o quitar algunos existentes, todo sin más rodeos que configurar el \eng{broker} acorde.
  \item \textbf{Escalable:} puede crecer en las distintas dimensiones del modelo \eng{scale cube} (ver \autoref{propuesta:escalabilidad}) según sea necesario.
  \item \textbf{Configurable:} su concepción está basada en la aceptación del cambio, y a tal fin debe poder ser modificada (o configurada) ante necesidades cambiantes.
  \item \textbf{Orientada a la \eng{performance}:} la introducción de diversas capas de caching, el balanceo de la carga de procesamiento entre diferentes instancias redundantes de los nodos, y la reescritura de la librería cliente para adoptar los beneficios que los estándares y el \gls{proto:http} caching están enfocados en reducir los tiempos de procesamiento y respuesta de los servicios, lo cual resulta en un mejor rendimiento general de la {\cloud}.
  \item \textbf{Basada en tecnologías Open Source:} siguiendo nuestros principios de trabajo, orientamos la elección de las herramientas utilizadas en todas las capas de la {\cloud} a productos Open Source, utilizadas por un número enorme de empresas y organismos, y que poseen comunidades sólidas que los respaldan y mantienen.
  \item \textbf{Basada en estándares abiertos:} al emplear estándares como \nameref{soa:tecnologias:json-api} y \gls{proto:http} caching, resulta más sencilla la integración de nuestro \eng{stack} de tecnologías con otros productos, e inclusive nos permite aprovechar herramientas desarrolladas por terceros para esos estándares sin necesidad de realizar implementaciones \textit{ad hoc}.
  \item \textbf{Hecha a medida:} se origina y desarrolla con la {\cloud} en mente, lo cual la hace altamente específica para los casos de uso que tendrá. Así como se basa en análisis e investigaciones del estado del arte en la materia, también está planteada a partir del aprendizaje de nuestros propios errores.
  \item \textbf{Simple:} intentamos incluir la cantidad justa de nodos para cumplir los roles necesarios en el diseño, sin agregar capas innecesarias ni eliminar puntos posibles de expansión al quitar algún nodo que habilite la redundancia.
  \item \textbf{Desacoplada:} cada \eng{service component} es una unidad lógica encargada de un conjunto concreto de tareas, cuyas interdependencias se manejan a través del \eng{broker} y no de forma directa, lo cual acoplaría un servicio a otro.
  \item \textbf{Robusta:} gracias a la redundancia y el desacoplamiento antes mencionados, el uso del \eng{broker} intermedio y las tecnologías empleadas, que han sido extensamente probadas en ambientes de producción.
  \item \textbf{Testeable:} su diseño descompuesto en capas y unidades independientes permite fácilmente aislar partes de la arquitectura para realizar pruebas sobre ellas.
\end{itemize}

En el siguiente apartado se presentan algunas tareas derivadas de la presente tesina para su posterior tratamiento.
