\newglossaryentry{term:antipatron} {
  name = {Antipatrón de diseño de software},
  description = {es un patrón de diseño que encamina el software a una situación desfavorable o problemática},
  sort = {antipatron}
}

\newglossaryentry{term:backend} {
  name = {\textit{backend}},
  description = {\textit{TODO}},
  sort = {backend}
}
\newglossaryentry{scm:git} {
  name = {\textit{git}},
  description = {https://git-scm.com},
  sort = {backend}
}

\newglossaryentry{term:rest} {
  name = {\textit{REST}},
  description = {es un estilo de arquitectura de sistemas distribuidos definida por Roy T. Fielding en su tesis Doctoral},
  sort = {rest}
}

\newglossaryentry{scm:subversion} {
  name = {\textit{subversion}},
  description = {http://subversion.apache.org},
  sort = {subversion}
}

\newglossaryentry{fw:symfony} {
  name = {\textit{symfony}},
  description = {es un \eng{framework} web desarrollado en el lenguaje PHP, que al momento de elaboración del presente trabajo se encuentra en su segunda versión mayor. A modo de referencia, en nuestros desarrollos siempre utilizamos la versión 1.x. – http://symfony.com},
  sort = {symfony}
}

\newacronym{api}{API}{\eng{Application Programming Interface}}
