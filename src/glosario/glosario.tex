\newpage
\newglossaryentry{term:antipatron} {
  name = {antipatrón de diseño de software},
  description = {es un patrón de diseño que encamina el software a una situación desfavorable o problemática},
  sort = {antipatrondedisenodesoftware}
}

\newglossaryentry{term:aplicacionessatelitales} {
  name = {aplicaciones satelitales},
  description = {término que hemos acuñado en nuestro trabajo diario para calificar a aquellas aplicaciones que utilizan la nube de servicios de la UNLP},
  sort = {aplicacionessatelitales}
}

\newglossaryentry{term:bloating} {
  name = {bloating},
  description = {exceso de código y funcionalidad a causa de la inclusión de lógica que al momento del diseño o desarrollo se considera que \textit{podría ser útil}, pero que en la práctica no se utiliza},
  sort = {bloating}
}

\newglossaryentry{term:captcha} {
  name = {captcha},
  description = {prueba automatizada utilizada para determinar cuándo el usuario es o no humano},
  sort = {captcha}
}

\newglossaryentry{term:documentacion-viviente} {
  name = {documentación viviente},
  description = {versión en línea de una documentación que puede utilizarse para probar aquello que ella describe. En el caso específico de las \glspl{acro:api} web, permite consumir \glspl{term:endpoint} de ejemplo para conocer qué parámetros admiten y qué respuestas ofrecen},
  sort = {documentacionviviente}
}

\newglossaryentry{term:endpoint} {
  name = {endpoint},
  description = {punto de conexión o acceso a un servicio},
  sort = {endpoint}
}

\newglossaryentry{scm:git} {
  name = {git},
  description = {\url{https://git-scm.com}},
  sort = {git}
}

\newglossaryentry{acro:hmac} {
  name = {HMAC},
  description = {\eng{Hash Message Authentication Code}, protocolo criptográfico para la firma de mensajes, definido en la RFC 2104. - \url{https://tools.ietf.org/html/rfc2104}},
  sort = {hmac}
}

\newglossaryentry{term:hypermedia} {
  name = {hypermedia},
  description = {forma básica de conexión de contenidos en la web, estableciendo vínculos lógicos a distintos niveles o hipervínculos entre los distintos medios que la componen},
  sort = {hypermedia}
}

\newglossaryentry{acro:jwt} {
  name = {JWT},
  description = {\eng{JSON Web Tokens}, estándar de intercambio de mensajes utilizando tokens de control definido en la RFC 7519. - \url{https://tools.ietf.org/html/rfc7519}},
  sort = {jwt}
}

\newglossaryentry{acro:jsonapi} {
  name = {JSON API},
  description = {\eng{JSON API}, es una especificación que indica como un cliente debe solicitar o modificar recursos y como el servidor debe responder a esas solicitudes. - \url{http://jsonapi.org/}},
  sort = {jsonapi}
}

\newglossaryentry{db:memcached} {
  name = {Memcached},
  description = {es un almacen de datos en memoria distribuido \eng{open source}, organizado en pares clave-valor que es habitualmente utilizado como cache compartida para evitar accesos a recursos más costosos del lado del servidor. - \url{http://memcached.org}},
  sort = {memcached}
}

\newglossaryentry{db:varnish} {
  name = {Varnish},
  description = {es un acelerador de aplicaciones web, también conocido como caché de proxy reverso HTTP, utilizado como cache compartida para evitar accesos a recursos el lado del servidor. - \url{https://www.varnish-cache.org}},
  sort = {varnish}
}

\newglossaryentry{db:nosql} {
  name = {NoSQL},
  description = {nueva tendencia en motores de bases de datos que están orientadas a mejorar problemas posibles del paradigma tradicional de las bases de datos relacionales. Como principio, son no relacionales, distribuidas y escalables horizontalmente. Algunos ejemplos son Hadoop, Cassandra, CouchDB y MongoDB},
  sort = {nosql}
}

\newglossaryentry{fw:rails} {
  name = {Ruby on Rails},
  description = {es un \eng{framework} hecho en Ruby para el desarrollo ágil de aplicaciones web. Es, de hecho, \textit{el} \eng{framework} web por excelencia del lenguaje. – \url{http://rubyonrails.org}},
  sort = {rubyonrails}
}

\newglossaryentry{db:redis} {
  name = {Redis},
  description = {es un almacen en memoria de datos de código abierto, que suele ser utilizado como \eng{cache}, base de datos de alto rendimiento (para conjuntos reducidos de datos) o cola de mensajes entre procesos distribuidos. - \url{http://redis.io}},
  sort = {redis}
}

\newglossaryentry{term:refactor} {
  name = {refactor},
  description = {técnica utilizada en el desarrollo de software en la que, luego de identificar puntos posibles de mejora para una pieza de software, se modifica su implementación interna manteniendo su interfaz externa y comportamiento},
  sort = {refactor}
}

\newglossaryentry{term:rest} {
  name = {REST},
  description = {es un estilo de arquitectura de sistemas distribuidos definida por Roy T. Fielding en su tesis Doctoral. Referirse a la \autoref{standard:rest} para mayores detalles.},
  sort = {rest}
}

\newglossaryentry{term:restful} {
  name = {RESTful},
  description = {término que se utiliza para denotar aquellas \glspl{acro:api} que cumplen, al menos parcialmente, con las propiedades de la arquitectura de sistemas distribuidos \gls{acro:rest}.},
  sort = {restful}
}

\newglossaryentry{fw:sinatra} {
  name = {Sinatra},
  description = {es un \gls{acro:dsl} para el desarrollo ágil de aplicaciones web implementado en el lenguaje Ruby. – \url{http://sinatrarb.com}},
  sort = {sinatra}
}

\newglossaryentry{scm:subversion} {
  name = {subversion},
  description = {\url{http://subversion.apache.org}},
  sort = {subversion}
}

\newglossaryentry{fw:symfony} {
  name = {symfony},
  description = {es un \eng{framework} web desarrollado en el lenguaje PHP, que al momento de elaboración del presente trabajo se encuentra en su segunda versión mayor. A modo de referencia, en nuestros desarrollos siempre utilizamos la versión 1.x. – \url{http://symfony.com}},
  sort = {symfony}
}

\newglossaryentry{ws:webservice} {
  name = {Web Service},
  description = {estándar desarrollado por la W3C que engloba diferentes tecnologías utilizadas para realizar comunicaciones máquina-máquina sobre el medio web - \url{http://www.w3.org/TR/ws-arch/\#whatis} y \url{http://www.w3.org/TR/ws-arch/\#technology}},
  sort = {webservice}
}

\newglossaryentry{term:uuid} {
  name = {Universally Unique Identifier (UUID)},
  description = {Es un número de 16 bytes que suele expresarse como 32 caracteres hexadecimales y 4 guiones, siguiendo el formato \texttt{0000000-0000-0000-0000-000000000000}. Tienen la característica de tener chances extremadamente bajas de generar colisiones cuando se los genera aleatoriamente.},
  sort = {uuid}
}

\newacronym{acro:acl}{ACL}{\eng{Access Control List}}
\newacronym{acro:api}{API}{\eng{Application Programming Interface}}
\newacronym{acro:corba}{CORBA}{\eng{Common Object Broker Architecture}}
\newacronym{acro:cors}{CORS}{\eng{Cross-Origin Resource Sharing}}
\newacronym{acro:dsl}{DSL}{\eng{Domain-Specific Language}}
\newacronym{acro:dcom}{DCOM}{\eng{Distributed Computing Object Model}}
\newacronym{acro:esb}{ESB}{\eng{Enterprise Service Bus}}
\newacronym{acro:etl}{ETL}{\eng{Extract, Transform and Load}}
\newacronym{acro:ftp}{FTP}{\eng{File Transfer Protocol}}
\newacronym{acro:hateoas}{HATEOAS}{\eng{Hypertext As The Engine Of Application State}}
\newacronym{lang:html}{HTML}{\eng{HyperText Markup Language}}
\newacronym{proto:http}{HTTP}{\eng{HyperText Transfer Protocol}}
\newacronym{lang:json}{JSON}{\eng{JavaScript Object Notation}}
\newacronym{acro:mvc}{MVC}{\eng{Model-View-Controller}}
\newacronym{acro:nfs}{NFS}{\eng{Network File System}}
\newacronym{acro:oai}{OAI}{\eng{Open API Initiave}}
\newacronym{acro:oasis}{OASIS}{\eng{Organization for the Advancement of Structured Information Standards}}
\newacronym{acro:rest}{REST}{\eng{REpresentational State Transfer}}
\newacronym{acro:rpc}{RPC}{\eng{Remote Procedure Call}}
\newacronym{acro:saml}{SAML}{\eng{Security Assertion Markup Language}}
\newacronym{acro:soa}{SOA}{\eng{Service-Oriented Architecture}}
\newacronym{ws:soap}{SOAP}{\eng{Simple Object Access Protocol}}
\newacronym{acro:spof}{SPOF}{\eng{Single Point Of Failure}}
\newacronym{acro:ssl}{SSL}{\eng{Secure Sockets Layer}}
\newacronym{acro:sso}{SSO}{\eng{Single Sign-On}}
\newacronym{acro:ttl}{TTL}{\eng{Time To Live}}
\newacronym{acro:uri}{URI}{\eng{Uniform Resource Identifier}}
\newacronym{acro:url}{URL}{\eng{Uniform Resource Locator}}
\newacronym{acro:urn}{URN}{\eng{Uniform Resource Name}}
\newacronym{acro:uddi}{UDDI}{\eng{Universal Description Discovery and Integration}}
\newacronym{ws:wsdl}{WSDL}{\eng{Web Services Description Language}}
\newacronym{lang:xml}{XML}{\eng{eXtensible Markup Language}}
\newacronym{lang:xsd}{XSD}{\eng{XML Schema}}
\newacronym{lang:yaml}{YAML}{\eng{Yet Another Markup Language}}
