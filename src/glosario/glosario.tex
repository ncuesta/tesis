\newglossaryentry{antipatron} {
  name = {Antipatrón de diseño de software},
  description = {es un patrón de diseño que encamina el software a una situación desfavorable o problemática}
}

\newglossaryentry{backend} {
  name = \textit{backend},
  description = {\textit{TODO}}
}
\newglossaryentry{git} {
  name = \textit{git},
  description = {https://git-scm.com}
}

\newglossaryentry{REST} {
  name = \textit{REST},
  description = {es un estilo de arquitectura de sistemas distribuidos definida por Roy T. Fielding en su tesis Doctoral}
}

\newglossaryentry{subversion} {
  name = \textit{subversion},
  description = {http://subversion.apache.org}
}

\newglossaryentry{symfony} {
  name = \textit{symfony},
  description = {es un \eng{framework} web desarrollado en el lenguaje PHP, que al momento de elaboración del presente trabajo se encuentra en su segunda versión mayor. A modo de referencia, en nuestros desarrollos siempre utilizamos la versión 1.x. – http://symfony.com}
}

\newacronym{api}{API}{\eng{Application Programming Interface}}
